\documentclass[12pt]{article}

\usepackage{amsmath,amssymb}
\usepackage{supertabular}
\usepackage{geometry}
\usepackage{ifthen}
\usepackage[usenames,dvipsnames,svgnames,table,x11names]{xcolor}
\usepackage{alltt}%hack


\newif\ifdraft
%\drafttrue
\draftfalse
\newcommand{\cjc}[1]{\ifdraft{\color{Blue}[\textbf{CJC}: #1]}\fi}

\include{plang}

\title{Policy Language: Theoretical Design Document}
\date{October 2018}

\renewcommand{\ottusedrule}[1]{\[#1\]\vspace{1em}}


%% \renewcommand{\ottgrammartabular}[1]
%%   {\begin{minipage}{\columnwidth}\begin{tabular}{ll}#1\end{tabular}\end{minipage} }
%% \renewcommand{\ottrulehead}[3]
%%   {$#1$  $#2$ & $#3$}
%% \renewcommand{\ottprodline}[6]
%%   { \quad $#1$ \ $#2$ & \quad $#3$  $#5$  $#6$}
%% \renewcommand{\ottinterrule}
%%   {\\[2.0mm]}
%% \renewcommand{\ottkw}{\mathsf}
%% \renewcommand{\ottdrule}[4][]{{\displaystyle\frac{\begin{array}{l}#2\end{array}}{#3}\;\ottdrulename{#4}}}

\begin{document}

\maketitle

This document gives a formal definition to a portion of the Draper/Dover Policy
Language (DPL).  Section~\ref{sec:syntax} gives the complete syntax for this portion
and describes its meaning informally.  Section~\ref{sec:opsem} gives an
operational semantics and provides some background information on design
decisions that went into the language.

\section{Syntax}
\label{sec:syntax}

Metavariables: $C$ ranges over constructor names, $x$ and $y$ range over variable
names, which include opgroups and pattern variables.

\subsection{Tags}

In DPL, metadata is always a set of {\em tags}.  Tags are
primitive units declared by each micropolicy.  For example, the RWX policy has
{\tt Rd}, {\tt Wr} and {\tt Ex} tags.  The metadata associated with a memory
word or register is a set of these tags.  For example, stack temporary memory
typically has the metadata {\tt \{Rd,Wr\}}, and executable memory typically has
the metadata {\tt \{Ex\}}.  We use the words ``metadata'' and ``tag set''
interchangeably, and take care to avoid using the word ``tag'' where
``metadata'' is meant.

In informal policy language documentation, we use the word ``values'' to refer
to what we call ``tags'' in this document.  We haven't made that switch here in
order to reserve the word ``value'' for use by the operational semantics.  At
the moment, however, the operational semantics don't need a notion of values, so
it might be better to unify the language.

\cjc{Most tags have no arguments (they are just a constructor $[[C]]$).  Tags
  with arguments are not quite right in the current version of this document.
  In particular, union and intersection are tricky for tags with arguments, but
  aren't commonly used.}

\bigskip

\ottgrammartabular{
\otttdecl\ottinterrule
\otttag\ottinterrule
\otttf\ottafterlastrule
}

\subsection{Policies}

The ``policies'' section of a DPL file contains a series of policy declarations,
assigning a name to a policy expression.  Policy expressions are built out of
one primitive ``rule'' form and three composition operators.

The $[[pexp1 | pexp2]]$ and $[[pexp1 ^ pexp2]]$ forms have the same operational
semantics but are intended to have different static semantics to help rule out
common errors (see Section~\ref{sec:static}) below.  In both cases, the left
policy is tried first.  If it has a rule that ``matches'' the current
instruction's metadata, that rule's answer is used.  otherwise, the policy on
the right is used.

The $[[pexp1 & pexp2]]$ form is used when combining two completely distinct
policies to be run in parallel.  This form is only allowed if $[[pexp1]]$ and
$[[pexp2]]$ use different tags.  When evaluating $[[pexp1 & pexp2]]$, each
policies sees metadata containing only its tags, as if it were running in
isolation.  An instruction is allowed only if both subpolicies allow it.

\bigskip

\ottgrammartabular{
\ottpdecl\ottinterrule
\ottpexp\ottafterlastrule
}

\bigskip

Rules comprise an opgroup $[[C]]$, a list of patterns, and result.  Opgroups
specify the subset of RISC-V instructions for which this rule applies and define
what memory and register metadata will be available in the rest of the rule (for
more details, see Section~\ref{sec:opgroups}).

A rule only ``matches'' the current instruction if that instruction is in
opgroup {\em and} all its patterns match.  Otherwise, the rule ``implicitly
fails'' and (typically) the next rule is tried.  If the rule does match, its
result either indicates an ``explicit'' policy failure, or provides instructions
on how to update the metadata (indicating that the instruction is accepted).
More on implicit and explicit failure in Section~\ref{sec:failure}.

\bigskip

\ottgrammartabular{
\ottrule\ottinterrule
\ottresult\ottafterlastrule
}

\bigskip

Pattern constraints have the form $[[x = tspat]]$.  Here, $[[x]]$ is a name
defined by the opgroup to refer to a particular memory address or register used
by this category of instructions.  $[[tspat]]$ expresses a constraint on the
corresponding tag set.

The simplest forms of patterns are $[[{t1,..,tk}]]$ and $[[ [tr1,..,trk] ]]$.
The former matches only if the corresponding tag set is exactly the set shown
(order does not matter).  The latter matches if each of the individual ``tag
requirements'' is true.  The tag requirement $[[+t]]$ indicates that a certain
tag must be present, while $[[-t]]$ indicates that a certain tag must be absent.
A tag $[[t]]$ by itself in this context is shorthand for $[[+t]]$.

The tag set may be bound to a name $[[x]]$ for use in the rule's results, using
the pattern $[[x]]$ which matches every tag set, or $[[x@tspat]]$ which matches
the same sets as $[[tspat]]$.  The pattern $\_$ binds no names and matches
every tag set.

\bigskip

\ottgrammartabular{
\otttspats\ottinterrule
\otttspat\ottinterrule
\otttr\ottafterlastrule
}

\bigskip

When a rule accepts an instruction, its conclusion indicates how to update the
relevant metadata with the form $[[x = tsexp]]$.  Here, $[[x]]$ is a name defined
by the opgroup to refer to a particular memory address or register that is
updated by this category of instructions.  $[[tsexp]]$ describes the new tag set
for this location.

The simplest form of expression are $[[{t1, .. , tk}]]$ and
$[[tsexp[tr1, .., trk] ]]$.  The former provides an exact tag set.  The latter
provides a list of modifications for another tag set $[[tsexp]]$ (typically a
variable bound by the rule's patterns).  These modifications have the form
$[[+t]]$ or $[[-t]]$ indicating that a tag should be added if absent or removed
if present.

\cjc{The union and intersection forms are a little messy.  If you try to union
  two sets containing two different colors in the heap policy, you get a dynamic
  error.}

\bigskip

\ottgrammartabular{
\otttsexps\ottinterrule
\otttsexp\ottafterlastrule
}


\section{Operational Semantics}
\label{sec:opsem}

\subsection{Design Decisions}

\subsubsection{Opgroups} 

\label{sec:opgroups}

In the implementation, opgroups serve two purposes: one
is to identify the instruction being executed, and the other is to map the
operand names used in the ISA documentation to the intuitive field names written
in our rules.  This is desirable for two reason: First, there are some classes
of instructions where the operands have different names in the ISA spec, but the
meaning of the instruction is similar enough that we want to write rules that
cover the whole class.  Second, it abstracts away from the ISA, potentially
making it possible to write policies that apply for multiple ISAs.

In the language we describe here, we use opgroups only to identify the
instruction, and ignore the question of hardware/ISA spec names.  Put another
way, before executing the semantics described here, it's necessary to first
translate from the ISA spec operand names to the rule names, and translate back
after.  This simplifies the semantics.

A more complete formalization of the language would include the ``opgroups''
section from the DPL files and construct the mapping from hardware names to
intuitive names accordingly.

\subsubsection{Failure}
\label{sec:failure}

At execution time, a policy can fail for two reasons:

\begin{itemize}
\item The policy specifies that the given set of tags is a violation, via the
  $\mathsf{fail}$ keyword.  We call this {\em explicit} failure.
\item The policy does not contain a rule that matches the given set of tags.
  We call this {\em implicit} failure.
\end{itemize}

Explicit and implicit failure have different meanings, especially when dealing
with incomplete pieces of policies.  A rule that fails explicitly indicates a
real policy violation, while it may fail implicitly because it is intended to be
composed with other rules to create a complete policy.

The operational semantics needs to be aware of this distinction.  Consider the
``pattern matching'' form of policy composition, $[[p1 ^ p2]]$  The intended
semantics here is that $[[p1]]$ applies if it has a rule for the given input,
and otherwise $[[p2]]$ applies.  If $[[p1]]$ contains an explicit failure, then
$[[p1 ^ p2]]$ should also fail.  But an implicit failure of $[[p1]]$ does not
trigger the failure of $[[p1 ^ p2]]$.

\subsection{Syntax}

We let $[[T]]$ range over sets of tags.  Tag environments $[[Del]]$ are maps
variables $[[x]]$ to tag sets $[[T]]$.  They are used in the input and output of
policy evaluation.  Variables in the domain are either provided by the execution
environment for the opgroup's memory and register metadata, or are created by
the rule by binding tag sets to variables with the $@$ form.  We write
$[[dom(Del)]]$ for the domain of a tag environment (that is, the set of variables
that are mapped to tag sets).

The empty tag environment (i.e., the tag environment with an empty domain) is
written $[[.]]$.  We write $[[Del [x |-> T] ]]$ for the tag environment that
maps $[[x]]$ to $[[T]]$ and otherwise has the same mappings as $[[Del]]$.  Finally, we
write $[[Del(x)]]$ for the tag set mapped to $[[x]]$ by $[[Del]]$ if
$[[x]] \in [[dom(Del)]]$.  It is undefined otherwise.

We use $R$ to denote the result of a policy evaluation, which is either implicit
failure $[[_|_]]$, explicit failure $[[fail]]$, or a new tag environment
$[[Del]]$.

\bigskip

\ottgrammartabular{
  \ottT\ottinterrule
  \ottDel\ottinterrule
  \ottR\ottafterlastrule
}

\bigskip

Two policies that refer to different sets of tags may be combined with the
$[[&]]$ operator, as in $[[p1 & p2]]$.  We implement this operator via a union
over policy results $[[R1 |_| R2]]$, which is defined as follows: If either
input is $[[fail]]$, it returns $[[fail]]$.  Otherwise, if either input is
$[[_|_]]$, it returns $[[_|_]]$.  Finally, n the $[[Del1 |_| Del2]]$ case, it is
defined to map each field name $[[x]] \in [[dom(Del1) \/ dom(Del2)]]$ to
$[[Del1(x) \/ Del2(x)]]$ if it appears in both domains, and to $[[Del1(x)]]$ or
$[[Del2(x)]]$ if it appears in only one domain.

\subsection{Pattern matching}

The most basic form of policy is the {\em rule clause}.
\[
   [[C(tspats -> result)]]
\]
Let's examine the pieces of this rule (see Section~\ref{sec:syntax} for a more
complete description:
\begin{itemize}
\item $C$ is an opgroup name.  Opgroups are tags, so this expresses a constraint
  that this rule only applies when the tag $C$ appears in the tag set for the
  current instruction.
\item $[[tspats]]$ is a list of pattern matching equations of the form
  $[[x = tspat]]$.  Here, $[[x]]$ is a {\em field name}: a field is an element of
  the system that can have an associated tag, including memory, registers, and
  the program counter.  $[[tspat]]$ is a pattern.  This pattern serves two
  purposes: (a) it can put some constraints on $[[x]]$'s tag set (for example, that
  it must include or not include a given tag), and (b) it can bind a name for
  $[[x]]$'s tag set so that it can be mentioned in the conclusion of the rule.
\item $[[result]]$ is either $[[fail]]$ (explicit failure), or it is a list of
  tag set definitions of the form $[[x = tsexp]]$.  As in $[[tspats]]$, $[[x]]$
  identifies a tag set to be updated.  The tag expression $[[tsexp]]$ defines the
  updated tag set for $x$ after the instruction executes.
\end{itemize}

Part of evaluating a rule clause against a tag environment $[[Del]]$ is checking
whether $[[tspats]]$ matches against $[[Del]]$, and computing a binding for
pattern variables if so.  The computed map from pattern variables to tag sets is
itself a tag environment $[[Del']]$.  Below, we define three functions that are
used in the dynamic semantics to compute this result:

\begin{itemize}
\item $[[contains(T,tr1,...,trk)]]$ checks whether $[[T]]$ satisfies the
  requirement expressed by each $[[tri]]$, returning a boolean result.

\item $[[match (T , Del, tspat)]]$ checks whether the pattern $[[tspat]]$
  matches against the tag set $[[T]]$.  If not, it returns $[[_|_]]$.  If so, it
  returns the result of updating $[[Del]]$ with any corresponding new bindings.
\item $[[matches (Del, tspats)]]$ checks every pattern in $[[tspats]]$ against
  the current tag environment $[[Del]]$, and
  collects the results.
\end{itemize}

\bigskip
\framebox{$[[contains(T,tr1,...,trk)]] = [[bool]]$}
\begin{align*}
  [[contains(T,.)]] &= [[true]]\\
  [[contains(T,t,trs)]] &= t \in [[T]] \wedge [[contains(T,trs)]]\\
  [[contains(T,+t,trs)]] &= t \in [[T]] \wedge [[contains(T,trs)]]\\
  [[contains(T,-t,trs)]] &= t \notin [[T]] \wedge [[contains(T,trs)]]
\end{align*}

\bigskip
\framebox{$[[match(T,Del,tspat)]] = [[R]]$}

\begin{align*}
  [[match(T,Del,_)]] &= [[Del]]\\
  [[match(T,Del,y)]] &= [[Del [ y |-> T] ]]\\
  [[match(T,Del,y@tspat)]] &=
      \begin{cases}
        [[Del'[y |-> T] ]] & \text{if $[[match(T,Del,tspat)]]$ = $[[Del']]$}\\
        [[_|_]] & \text{if $[[match(T,Del,tspat)]]$ = $[[_|_]]$}
      \end{cases}
     \\
  [[match(T,Del,{t1,...,tk})]] &=
      \begin{cases}
        [[Del]] & \text{if } [[T]] = [[{t1,...,tk}]]\\
        [[_|_]] & \text{otherwise}
      \end{cases}
     \\
  [[match(T,Del,[tr1,...,trk])]] &=
      \begin{cases}
        [[Del]] & \text{if } [[contains(T,tr1,...,trk)]] = [[true]]\\
        [[_|_]] & \text{if } [[contains(T,tr1,...,trk)]] = [[false]]
      \end{cases}
\end{align*}

\ottdefnmsev

\bigskip

Finally, we also describe how to evaluate a collection of tag set expressions,
formalized as a judgement $[[Del1 |- tsexps => Del2]]$.  This definition makes
use of a tag environment $[[Del1]]$ that provides values for any tag set
variables in the expressions and is typically the result of the $[[match]]$
function defined above.  It relies on a secondary judgement,
$[[Del1 |- tsexp => T]]$, which evaluates an individual tag set expression.  There
is also a helper function, $[[update(T,tr1,...,trk)]]$, which applies the
modification denoted by each $[[tri]]$ to the tag set $[[T]]$

\bigskip

\framebox{$[[update(T,tr1,...,trk)]] = [[T']]$}
\begin{align*}
  [[update(T,.)]] &= [[T]]\\
  [[update(T,t,trs)]] &= [[update(T,trs)]] \cup \{t\}\\
  [[update(T,+t,trs)]] &= [[update(T,trs)]] \cup \{t\}\\
  [[update(T,-t,trs)]] &= [[update(T,trs)]] \setminus \{t\}
\end{align*}

\ottdefntseev

\ottdefntsesev


\subsection{Static semantics}
\label{sec:static}

Our intent is to add a type system that statically rules out common classes of
errors.  For example:

\begin{itemize}
\item The $[[p1 | p2]]$ operator is intended to be used only in situations where
  $[[p1]]$ and $[[p2]]$ don't ``overlap''.  That is, for any $[[Del]]$, at least
  one of $[[p1]]$ or $[[p2]]$ should result in implicit failure.

\item The $[[p1 & p2]]$ operator is intended to be used only when $[[p1]]$ and
  $[[p2]]$ are defined in distinct modules that refer to distinct sets of tags.
  That is, its use is in composing disjoint policies, not pieces of one policy.
\end{itemize}

We don't currently have a type system or notion of modules in this semantics,
though, so these and other statically-detectable errors are not caught at
compile time and may result in buggy policies.

\subsection{Policy Evaluation}

We write $[[Del |- p => R]]$ to indicate that the policy $[[p]]$ has result
$[[R]]$ when evaluated in tag environment $[[Del]]$.  If the policy accepts
this tag environment, the output will be a set of updated tags.  In the case of
an explicit failure, the output will be $[[fail]]$.  In the case of implicit
failure, the output will be $[[_|_]]$.  See Section~\ref{sec:failure} for more
on this distinction.
  
This judgement is defined by the ``big-step'' operational semantics that follows.
One point of interest is the evaluation rules for $[[|]]$ and $[[^]]$.  The
$[[|]]$ operator is evaluated as if it were $[[^]]$.  This makes sense, because
if the non-overlapping check for $[[|]]$ has succeeded, then it has the same
behavior $[[^]]$.  The rules for $[[^]]$ enforce the intent that the right
policy applies only if the left doesn't.  In general, the operational semantics
assume that static checks will rule out obvious errors (even though we haven't
implemented those checks yet).

Note that the rules make explicit the implementation detail that opgroups are
just tags in the current instruction's metadata.  The definition of the
$[[matches]]$ judgement allows for multiple patterns matches on the same tag set
name.  It would be good to find another way to express this, or to make opgroups
more explicit in general.

\cjc{This definition ignores the possibility of bound names for policies.  This
  isn't hard, I just need to add another environment/context.}

\bigskip

\ottdefnpev


\end{document}

